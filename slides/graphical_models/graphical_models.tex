% Input common header
\documentclass[xcolor=dvipsnames]{beamer}

\usecolortheme[named=Blue]{structure}
\setbeamertemplate{itemize items}[circle]

\usepackage{smartdiagram}


\author{Dr. Paul Larsen}
\date{\today}



\title{Graphical Models for Risk}
\begin{document}
\maketitle

\begin{frame}
\frametitle{d-separation}
\emph{Definition} Let $X$, $Y$, and $Z$ be three disjoint subsets of nodes in a directed acyclic graph, $G$, and let $p$ be any path betweeen a node in $X$ and one in $Y$, where by `path'we mean a succession of arcs, independent of direction. Then $Z$ is said to block $p$ if there is a node $w$ on $p$ satisfying one of the following two conditions

\begin{enumerate}
\item $w$ has converging arrows along $p$, and neither $w$ nor any of its descendents are in $Z$, or
\item $w$ does not have converging arrows along $p$, and $w$ is in $Z$.
\end{enumerate}

Further, $Z$ is said to $d-separate$ $X$ from $Y$ in $G$, denoted 

\begin{equation*}
    (X \ci Y | Z)_G
\end{equation*}

if and only if $Z$ blocks every path from a node in $X$ to a node in $Y$.
\end{frame}


\begin{frame}
\frametitle{d-separation}
\framesubtitle{Intuitions}
From \href{https://www.andrew.cmu.edu/user/scheines/tutor/d-sep.html}{Richard Scheines tutorial}:\newline

\begin{quote}
Roughly, two variables $X$ and $Y$ are independent conditional on $Z$ if knowledge of $X$ gives you no extra information about $Y$ once you have knowledge about $Z$. In other words, once you know $Z$, $X$ adds nothing to what you know about $Y$.
\end{quote}
\end{frame}

\begin{frame}
    \frametitle{d-separation}
    \framesubtitle{Alternative defintion}
    From \href{https://www.andrew.cmu.edu/user/scheines/tutor/d-sep.html}{Richard Scheines tutorial}
    
    \begin{quote}
    Roughly, two variables $X$ and $Y$ are independent conditional on $Z$ if knowledge of $X$ gives you no extra information about $Y$ once you have knowledge about $Z$. In other words, once you know $Z$, $X$ adds nothing to what you know about $Y$.
    \end{quote}
\end{frame}

\begin{frame}
    \frametitle{d-separation}
    \framesubtitle{Examples}

    \begin{figure}[ht]
        \begin{center}
          \begin{tabular}{cc}
            (i)
            \begin{tikzpicture}
                                % Define nodes
                \node[obs]                               (c) {$c$};
                \node[obs, above=of c, xshift=-1.2cm] (a) {$a$};
                \node[obs, above=of c, xshift=1.2cm]  (b) {$b$};
        
                \edge {a} {c} ; %
                \edge {c} {b} ;
              
              \end{tikzpicture} &
              (ii)
              \begin{tikzpicture}

                % Define nodes
                \node[obs]                               (c) {$c$};
                \node[obs, above=of c, xshift=-1.2cm] (a) {$a$};
                \node[obs, above=of c, xshift=1.2cm]  (b) {$b$};
        
                \edge {b} {c} ; %
                \edge {c} {a} ; %
              
              \end{tikzpicture}
          \end{tabular}
          \begin{tabular}{cc}
            (iii)
            \begin{tikzpicture}
                                % Define nodes
                \node[obs]                               (c) {$c$};
                \node[obs, above=of c, xshift=-1.2cm] (a) {$a$};
                \node[obs, above=of c, xshift=1.2cm]  (b) {$b$};
        
                \edge {c} {a} ; %
                \edge {c} {b} ;
              
              \end{tikzpicture} &
              (iv)
              \begin{tikzpicture}

                % Define nodes
                \node[obs]                               (c) {$c$};
                \node[obs, above=of c, xshift=-1.2cm] (a) {$a$};
                \node[obs, above=of c, xshift=1.2cm]  (b) {$b$};
        
                \edge {a} {c} ; %
                \edge {b} {c} ; %
              
              \end{tikzpicture}
          \end{tabular}
        \end{center}
        \caption{Examples.}
      \end{figure}

\end{frame}


\end{document}