% Input common header
\documentclass[xcolor=dvipsnames]{beamer}

\usecolortheme[named=Blue]{structure}
\setbeamertemplate{itemize items}[circle]

\usepackage{smartdiagram}


\author{Dr. Paul Larsen}
\date{\today}



\title{Correlation and Causality}
\begin{document}
\maketitle

\begin{frame}
\frametitle{Why causality matters}
    \centering
    \begin{tikzpicture}
        \node[inner sep=0pt] (proxy_caption) at (0,0){
            Because correlation is a proxy.
        };

        \node[inner sep=0pt, below=0.5 of proxy_caption] (proxy) at (0,0) {
            \fbox{
                \includegraphics[width=.75\textheight]{graphics/spiders_spelling}
            }
        };
    \end{tikzpicture}

\cite{spurious-spiders-spelling}
\end{frame}

\begin{frame}
\frametitle{Why causality matters}
\centering
\begin{tikzpicture}

    \node[inner sep=0pt] (med_caption) at (0,0) {
            Because A / B testing is not always possible.
    };
    \node[inner sep=0pt, below=0.5cm of med_caption] (med)  {
        \fbox{
            \includegraphics[width=.9\textheight]{graphics/mediterranean}
        }
    };

\end{tikzpicture}

\cite{estruch2013primary}
\end{frame}


\begin{frame}
    \frametitle{Simpson's paradox: cautionary tales}
    Simpson's paradox: a phenomenon in probability and statistics in which a trend appears disappears or reverses depending on grouping of data. \cite{simpson-wikipedia}, \cite{pearl2016causal} \newline

    Example: University of California, Berkeley 1973 admission figures\newline

    \begin{figure}[ht]
        \includegraphics[height=0.15\textheight]{graphics/berkeley}\newline
        \cite{freedman1998statistics}
    \end{figure}
    \begin{figure}[ht]
        \includegraphics[height=0.3\textheight]{graphics/berkeley_later}\newline
        \cite{Bickel398}
    \end{figure}
\end{frame}

\begin{frame}{A brief, biased history of causality}
    \begin{itemize}
        \item Aristotle, 384 - 322 BC
        \item Isaac Newton, 1643 - 1727 AD
        \item David Hume, 1711 - 1776 AD
        \item Francis Galton, 1822 - 1900 AD, Karl Pearson, 1857 - 1936 AD
        \item Judea Pearl, b. 1936 AD
    \end{itemize}
\end{frame}


\begin{frame}{Counterfactuals and causality}
    Ideal: Intervention + \href{https://en.wikipedia.org/wiki/Multiverse}{Multiverse} $\rightarrow$ Causality\newline

    Examples:
    \begin{itemize}
        \item Medical treatment (e.g. \href{https://en.wikipedia.org/wiki/Simpson\%27s_paradox\#Kidney_stone_treatment}{kidney stone treatment})
        \item Social outomes (e.g. \href{https://en.wikipedia.org/wiki/Simpson\%27s_paradox\#UC_Berkeley_gender_bias}{university admissions})
        \item Business outcomes (e.g. \href{https://en.wikipedia.org/wiki/Click-through\_rate}{click-through rate}, hit rate)\newline
    \end{itemize}

    In-practice:
    \begin{itemize}
        \item Correlation: approximate multiverse by comparing intervention at $t$ to result at $t-1$
        \item Random population: approximate multiverse by splitting sample well
        \item A / B testing: random populations A / B + intervention in one
    \end{itemize}
\end{frame}

\begin{frame}{Counterfactual example: hit rate for insurance}
    Variables:
    \begin{itemize}
        \item producttype: Client line of business
        \item days: Number of days to generate quote
        \item rating: Binary indication of client risk
        \item hit: Binary, 1 for success (binding the quote), 0 for failure\newline
    \end{itemize}

    Fake data:\newline\newline
    \begin{tabular}{lrrr}
\toprule
product\_type &  days &  rating &  hit \\
\midrule
    property &     3 &       1 &    0 \\
   liability &     1 &       0 &    0 \\
   financial &     0 &       1 &    0 \\
   liability &     3 &       0 &    0 \\
   liability &     0 &       0 &    1 \\
\bottomrule
\end{tabular}
\newline
\end{frame}


\begin{frame}{Counterfactual example: hit rate for insurance}
    Variables:
    \begin{itemize}
        \item producttype: Client line of business
        \item days: Number of days to generate quote
        \item rating: Binary indication of client risk
        \item hit: Binary, 1 for success (binding the quote), 0 for failure\newline
    \end{itemize}

    \begin{figure}[ht]
        \includegraphics[height=0.4\textheight]{graphics/hits}
    \end{figure}
\end{frame}

\begin{frame}{Non-counterfactual approach: condition and query}

    Goal: estimate effect of $\textrm{days}$ on $\textrm{hit}$.\newline

    Calculate
    \begin{itemize}
        \item $P(\textrm{hit}=1 | \textrm{days} = 0) - P(\textrm{hit}=1 | \textrm{days} = 1)$,
        \item $P(\textrm{hit}=1 | \textrm{days} = 1) - P(\textrm{hit}=1 | \textrm{days} = 2)$,
        \item $\ldots$ \newline
    \end{itemize}

    From exercise Jupyter notebook:\newline
    \begin{tabular}{lr}
\toprule
{} &       hit \\
days &           \\
\midrule
0    &  0.532706 \\
1    &  0.442064 \\
2    &  0.330519 \\
3    &  0.174006 \\
\bottomrule
\end{tabular}

\end{frame}


\begin{frame}{The Structural Causal Model}
    The definitions in following slides are from \cite{pearl2007mathematics}, \cite{pearl2016causal}.
    \begin{definition}
        A \emph{structural causal model} $M$ consists of two sets of variables $U, V$ and a set of functions $F$, where

        \begin{itemize}
            \item $U$ are considered \emph{exogenous}, or background variables,
            \item $V$ are the \emph{causal} variables, i.e. that can be manipulated, and
            \item $F$ are the functions that represent the process of assigning values to elements of $V$ based on other values in $U, V$, e.g. $v_i = f(u, v)$.
        \end{itemize}

        We denote by $G$ the graph induced on $U, V$ by the functions $F$, and call it the \emph{causal graph} of $(U, V, F)$.\newline
    \end{definition}

    Hit rate example: $U = \{\textrm{producttype}, \textrm{rating}\}$, $V= \{\textrm{days}, \textrm{hit}\}$, $F\leftrightarrow$ sample from conditional probabilty tables in directed graphical model.
\end{frame}

\begin{frame}{Formalizing interventions: the intuition of ``do"}
    For business application, quantity of interest is not $P(\textrm{hit}=1 | \textrm{days}=d)$, but intervention $$P(\textrm{hit}=1 | \jpdo(\textrm{days}=d))$$.\newline
    \begin{figure}[ht]
        \includegraphics[height=0.4\textwidth]{graphics/do_days}
    \end{figure}
\end{frame}

\begin{frame}{Formalizing interventions: the intuition of ``do"}
    For business application, quantity of interest is effect of intervention / counterfactual %$P(\textrm{hit}=1 | \textrm{days}=d)$, but intervention $$P(\textrm{hit}=1 | \jpdo(\textrm{days}=d))$$.\newline
    \begin{columns}[T] % align columns
        \begin{column}{.48\textwidth}
        Not $P(\textrm{hit}=1 | \textrm{days}=d)$\newline
        \begin{figure}[ht]
            $G = $ \includegraphics[height=0.55\textwidth]{graphics/given_days}
        \end{figure}
        \end{column}%
    %    \hfill%
        \begin{column}{.48\textwidth}
            but $P(\textrm{hit}=1 | \jpdo(\textrm{days}=d))$\newline
                \begin{figure}[ht]
                $G' = G_{\underline{\textrm{days}}} =$
                \includegraphics[height=0.55\textwidth]{graphics/do_days}
            \end{figure}
        \end{column}%
    \end{columns}
\end{frame}

\begin{frame}
\frametitle{Formalizing interventions: the intuition of ``do"}
	First, find quantities unchanged between $G$ and $G' = G_{\underline{\textrm{days}}}$
         \begin{figure}[ht]
             \includegraphics[height=0.5\textheight]{graphics/do_days}
        \end{figure}
        \vspace{-0.5cm}
        \begin{align}
        P_{G'}( & \textrm{producttype} = p, \textrm{rating} = r) \nonumber \\
        & = P_G( \textrm{producttype} = p, \textrm{rating} = r) \\
        P_{G'}(&\textrm{hit}=1 | \textrm{producttype} = p, \textrm{rating} = r) \nonumber  \\
        & = P_G(\textrm{hit}=1 | \textrm{producttype} = p, \textrm{rating} = r)
        \end{align}
\end{frame}


\begin{frame}{Formalizing interventions: the intuition of ``do"}
    \vspace{-1.5cm}
    \begin{tikzpicture}
        \node[] (q_0) at (0, 0) { };
        \node[] (q_1) [right=9cm of q_0]{
            \includegraphics[width=.35\textwidth]{graphics/do_days}
        };
    \end{tikzpicture}
    \vspace{-2.5cm}
    \begin{align*}
        P(&\textrm{hit}=1 | \jpdo(\textrm{days})=d) \\
        & = P_{G'}(\textrm{hit}=1 | \textrm{days}=d) \textrm{, by definition} \\
        & = \sum_{p, r} P_{G'}(\textrm{hit}=1 | \textrm{days}=d, \textrm{producttype} = p, \textrm{rating} = r) \\
        & \qquad  \qquad P_{G'}(\textrm{producttype} = p, \textrm{rating} = r | \textrm{days}=d)\textrm{, by total probability} \\
        & = \sum_{p, r} P_{G'}(\textrm{hit}=1 | \textrm{days}=d, \textrm{producttype} = p, \textrm{rating} = r) \\
        & \qquad  \qquad P_{G'}(\textrm{producttype} = p, \textrm{rating} = r), \textrm{ by substitution} \\
        & =  \sum_{p, r} P_{G}(\textrm{hit}=1 | \textrm{days}=d, \textrm{producttype} = p, \textrm{rating} = r) \\
        & \qquad  \qquad P_{G}(\textrm{producttype} = p, \textrm{rating} = r), \textrm{ our \emph{adjustment} formula}
    \end{align*}
    References: \cite{pearl2016causal}, \cite{chmp}
\end{frame}

\begin{frame}{Causal hit rate}
    Typical quantity of interest: \emph{average treatment effect} or \emph{ATE}
    \begin{columns}[T] % align columns
        \begin{column}{.48\textwidth}
            $P(\textrm{hit}=1 | \textrm{days}=d)$\newline

            \begin{tabular}{lr}
\toprule
{} &       hit \\
days &           \\
\midrule
0    &  0.532706 \\
1    &  0.442064 \\
2    &  0.330519 \\
3    &  0.174006 \\
\bottomrule
\end{tabular}
\newline\newline

            Example ATE: \newline
            $P(\textrm{hit}=1 | \textrm{days} = 2) $\newline
            $\quad - P(\textrm{hit}=1 | \textrm{days} = 3) \approx  16\%$
        \end{column}%
        \begin{column}{.48\textwidth}
            $P(\textrm{hit}=1 | \jpdo(\textrm{days}=d))$\newline

            \begin{tabular}{lr}
\toprule
{} &      prob \\
days &           \\
\midrule
0    &  0.549247 \\
1    &  0.410495 \\
2    &  0.292335 \\
3    &  0.215497 \\
\bottomrule
\end{tabular}
\newline \newline
            Example causal ATE: \newline
            $P(\textrm{hit}=1 |\jpdo(\textrm{days} )= 2)$\newline
            $\quad - P(\textrm{hit}=1 | \jpdo(\textrm{days}) = 3) \approx 8\%$
        \end{column}%
    \end{columns}

\end{frame}

\begin{frame}{Judea Pearl's Rules of Causality}

    Let $X$, $Y$ , $Z$ and $W$ be arbitrary disjoint sets of nodes in a DAG $G$. Let $G_\underline{X}$ be the graph obtained by removing all arrows pointing into (nodes of) $X$.
    Denote by $G_{\overline{X}}$ the graph obtained by removing all arrows pointing out of $X$. If, e.g. we remove arrows pointing out of $X$ and into $Z$, we the resulting graph is denoted by $G_{\underline{X} \overline{Z}}$

    Rule 1: Insertion / deletion of observations
    \begin{equation*}
        P(y | \jpdo(x), z, w) = P(y | \jpdo(x), w) \textrm{ if } (Y \ci Z | X, W)_{G_{\overline{X}}}
    \end{equation*}

    Rule 2: Action / observation exchange
    \begin{equation*}
        P(y | \jpdo(x), \jpdo(z), w) = P(y | \jpdo(x), z, w) \textrm{ if } (Y \ci Z | X, W)_{G_{\overline{X} \underline{Z}}}
    \end{equation*}

    Rule 3: Insertion / deletion of actions
    \begin{equation*}
        P(y | \jpdo(x), \jpdo(z), w) = P(y | \jpdo(x), w) \textrm{ if } (Y \ci Z | X, W)_{G_{\overline{X} \overline{Z(W)}}},
    \end{equation*}

    where $Z(W)$ is the set of $Z$-nodes that are not ancestors of any $W$-node in $G_\underline{X}$.

\end{frame}


\begin{frame}{Special cases of the causal rules}

    By judicious setting of sets of nodes to be empty, we obtain some useful corollaries of the causal rules.
    \newline

    Rule 1': Insertion / deletion of observations, with $W = \emptyset$
    \begin{equation*}
        P(y | \jpdo(x), z) = P(y | \jpdo(x)) \textrm{ if } (Y \ci Z | X)_{G_{\overline{X}}}
    \end{equation*}

    Rule 2': Action / observation exchange, with $X = \emptyset$
    \begin{equation*}
        P(y | \jpdo(z), w) = P(y | z, w) \textrm{ if } (Y \ci Z | W)_{G_{ \underline{Z}}}
    \end{equation*}

    Rule 3': Insertion / deletion of actions, with $X, W = \emptyset$
    \begin{equation*}
        P(y | \jpdo(z)) = P(y) \textrm{ if } (Y \ci Z )_{G_{\overline{Z}}}
    \end{equation*}

    \onslide<2->\textcolor{blue}{$\implies$ d-separation + causal rules = \emph{adjustment formulas}: $\jpdo$ queries as normal queries.}
\end{frame}

\begin{frame}[allowframebreaks]{References}
    \setbeamertemplate{bibliography item}[text]
    \bibliographystyle{amsalpha}
    \bibliography{../references.bib}
\end{frame}

\end{document}